\documentclass[12pt]{article}

\usepackage{geometry}
\geometry{a4paper, margin=1in}
\usepackage[utf8]{inputenc}
\usepackage[T1]{fontenc}
\usepackage{textcomp}
\usepackage[english]{babel}
\usepackage{amsmath, amssymb}
\usepackage{amsthm}

\newtheorem*{remark}{Remark}

\author{Maximo Fernandez}
\title{PDEs and Functional Analysis\\ Lecture 1}
\date{
    2022-08-15
}

\begin{document}
\maketitle
\section{Overview}
\label{sec:overview}
\subsection{Outline}
\label{sub:outline}
This course will cover PDEs and Functional Analysis. In particular:
\begin{enumerate}
    \item PDEs
    \begin{enumerate}
        \item First Order PDEs (Linear and Nonlinear)
        \item Second Order PDEs
        \begin{enumerate}
            \item Elliptic (Laplace's Equation)
            \item Parabolic (Heat Equation)
            \item Hyperbolic (Wave Equation)
        \end{enumerate}
    \end{enumerate}
    \item Functional Analysis
        \begin{enumerate}
            \item Hilbert Spaces
            \item Banach Spaces
            \item Fourier Transforms
        \end{enumerate}
\end{enumerate}
\subsection{Resources}
\label{sub:resources}
\textbf{Books}:
\\
\emph{Partial Differential Equations (Evans)}
\\
\emph{Functional Analysis and Partial Differential Equations (Brezis)}

\pagebreak
\section{First Order PDEs}
\label{sec:first_order_pdes}
\subsection{Introduction}
\label{sub:introduction}
Let $b\in\mathbb{R}^{n}$ and consider
\begin{equation}
u_t+b\cdot \nabla u=0   
\end{equation}
Where:
\begin{itemize}
    \item $u:\mathbb{R}\times\mathbb{R}^{n}\to \mathbb{R}$
    \item $u(t,x)\in\mathbb{R} \forall\, t\ge 0$
    \item $\nabla =(\frac{\partial}{\partial x_1},...,\frac{\partial}{\partial x_n})$
    \item $u_t=\frac{\partial }{\partial t}u$
\end{itemize}
Concretely, the equation becomes:
\begin{equation}
    \frac{\partial u}{\partial t}+\sum_{i=1}^n b_i\frac{\partial u}{\partial x_i}=0 
\end{equation}
Similarly, it can be written as:
\begin{equation}
    (u_t, \nabla u)\cdot(1,b)=0
\end{equation}

(1), (2) and (3) are equivalent, and are called \emph{Transport Equations} (of the first order). 
The equations are clearly homogeneous and of constant coefficients. 
To solve the transport equation, we are looking for a $u\in\mathcal{C}^{1}$ which satisfies the partial differential equation. 
The equation is said to be free, and has no boundary condition.
\\\\
The standard way to proceed is to:
\begin{enumerate}
    \item Look for special solutions (explicit)
    \item Review the existence of a solution in some class (the larger the class, the better)
    \item Review uniqueness of a solution (the smaller the class, the better)
    \item Review the regularity\footnote[1]{Typically means high integrability or high differentiability, but generally serves as a measure of having desirable qualities} of a solution (typically the difficult point)
\end{enumerate}
In order to review the existence of a solution, functional analysis is typically employed. In particular, Sobolov spaces and distributions are of importance.\\\\
Consider (1) once again. The equation is saying that the derivative along the $(1,b)$ vector is 0:
\begin{gather*}
    
\end{gather*}
\end{document}
