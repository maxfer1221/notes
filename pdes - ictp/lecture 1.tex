\documentclass[12pt]{article}

\usepackage{geometry}
\geometry{a4paper, margin=1in}
\usepackage[utf8]{inputenc}
\usepackage[T1]{fontenc}
\usepackage{textcomp}
\usepackage[english]{babel}
\usepackage{amsmath, amssymb}
\usepackage{amsthm}

\newtheorem*{remark}{Remark}
\newtheorem{example}{Example}[subsection]
\newtheorem{lemma}{Lemma}[section]

\author{Maximo Fernandez}
\title{Elliptic Partial Differential Equations\\ Lecture 1}
\date{
    2022-08-11
}

\begin{document}
\maketitle
\section{Classical Form}
\label{sec:classical_form}
Consider the following problem: 
Given $\Omega\subseteq\mathbb{R}^{n},n \ge 1, \Omega$ open, with
\[
    f:\Omega\to\mathbb{R}\text{, continuous}
\] 
\[
    g:\partial\Omega\to\mathbb{R}\text{, continuous}
\] 
Solve the following system:
\[
\begin{cases}
    u\in\mathcal{C}^2(\Omega)\cap\mathcal{C}^{0}(\overline{\Omega})\\
    -\Delta u = f & \text{in }\Omega\\
    u = g & \text{on }\partial\Omega
\end{cases}
\] 
Where $\Delta$ is the laplace operator: $\Delta u(x):=\text{div}(\nabla u(x)) $.
That is:\\
\\
$u$ is twice differentiable on $\Omega$, continuous on it's boundary, and\\
\[
\begin{cases}
    -\Delta u(x) = f(x) & \forall\, x\in \Omega\\
    u(x) = g(x) & \forall\, x\in\partial\Omega
\end{cases}
\]
where the first condition is called the \emph{Poisson Equation}, and the second condition is called the \emph{Dirichlet Boundary Condition}.

Our interests are in the existence of $u$, the uniqueness of $u$, and the regularity of $u$. These characteristics are found through our initial data $\Omega$, $f$, and $g$.
\subsection{Laplace's Equation}
\label{sub:laplace_s_equation}
Consider the above problem with the initial condition $f=0$ :
\[
\begin{cases}
    u\in\mathcal{C}^2(\Omega)\cap\mathcal{C}^{0}(\overline{\Omega})\\
    -\Delta u = 0 & \text{in }\Omega\\
    u = g & \text{on }\partial\Omega
\end{cases}
\]
The new PDE, $-\Delta u = 0$, is called \emph{Laplace's Equation}. Regularity is especially of interest with Laplace's Equation, as smoothness can be much higher than simply $u\in \mathcal{C}^2(\Omega)$
\section{Semiclassical Form}
Consider $\Omega$ bounded, $g\in\text{Lip}(\partial\Omega)$, that is, $g$ is Lipschitz continuous. Now let
 \[
     f(\xi)=\frac{|\xi|^2}{2}\quad\forall\,\xi\in\mathcal{R}
\]
And consider the \emph{Dirichlet Functional} 
\[
    F(u)=\int_\Omega f(\nabla u)\,dx=\frac{1}{2}\int_\Omega |\nabla u|^2\,dx\quad \forall u\in\text{Lip}(\Omega)
\] 
\begin{remark}
    $u\in\text{Lip}(\Omega)\implies u$ is differentiable almost everywhere in $\Omega$, so $|\nabla u|\in L^{\infty}(\Omega)$, implying the Dirichlet Functional is well defined.
\end{remark}
\label{sec:semiclassical_form}
The Semiclassical Approach is to minimize the Dirichlet Functional:
\[
    \inf\{F(u)|u\in\text{Lip}(\Omega), u=g \text{ on }\partial\Omega\}
.\] 
\begin{lemma}
Suppose that $u$ is a solution of the Semiclassical Approach, that is:
\[
    \frac{1}{2}\int_\Omega |\nabla u|^2\,dx= \inf\{F(u)|u\in\text{Lip}(\Omega), u=g \text{ on }\partial\Omega\}
\] 
And suppose $u\in\mathcal{C}^2(\Omega)\cap\mathcal{C}^{0}(\overline{\Omega})$, then $u$ solves the Classical Form with $f=0$ (Laplace's Equation).
\end{lemma}
\begin{proof}
    Take $\phi\in\text{Lip}(\Omega)$ and suppose $\phi$ has compact support in $\Omega$, that is, $\phi$ is 0 in the boundary of $\Omega$\footnote{Why? Look up compact support later.}: $\phi\in\text{Lip}_c(\Omega)\implies f(\partial\Omega)=\{0\}$. Further, let $\lambda\in\mathbb{R}$, and consider $u+\lambda\phi$:
    \[
        u+\lambda\phi\in\text{Lip}(\Omega)
        \text{ and }
        u+\lambda\phi=g \text{ on } \partial\Omega
\]
We can also compare $u+\lambda\phi$ to $u$ since they are both Lipschitz. Since $u$ solves the Semiclassical Approach:
\[
    F(u)\le F(u+\lambda\phi)\implies h(\lambda):=F(u+\lambda\phi)\text{ has minimum }\lambda=0
\]
Thus, $h'(0)=0$. Computing $h'$:
\[
    h'(\lambda)=
    \frac{d}{d\lambda }F(u+\lambda\phi)=
    \frac{d}{d\lambda } \frac{1}{2}
    \int_\Omega |\nabla u + \lambda\nabla \phi|^2\,dx
\]
\[
    \frac{1}{2}|\nabla u +\lambda\nabla \phi|^2=
    \frac{1}{2}\frac{d}{d\lambda}\langle\nabla u + \lambda\nabla \phi, \nabla u + \lambda\nabla \phi\rangle=\langle\nabla u+\lambda\nabla \phi,\nabla \phi\rangle
\]
\[
    \frac{d}{d\lambda}\frac{1}{2}\int_\Omega |\nabla u + \lambda\nabla \phi|^2\,dx=\int_\Omega \langle\nabla u+\lambda\nabla \phi, \nabla \phi\rangle\,dx
\] 
\[
h'(0)=\int_\Omega\langle\nabla u,\nabla \phi\rangle\,dx
\] 
Thus, $\int_\Omega\langle\nabla u,\nabla \phi\rangle\,dx=0 \quad\forall\phi\in\text{Lip}_c(\Omega)$. This is a weak formulation of the Classical Form. Integrating $h'(0)$ by parts\footnote{No clue how this works but I'll write an appendix to explain it eventually}:
\[
\int_\Omega\langle\nabla u, \nabla \phi\rangle\,dx=
-\int_\Omega\Delta u\phi\,dx=0\quad \forall \phi\in\text{Lip}_c(\Omega)
\] 
Sinc this is true for any $\phi$, $\Delta u$ must be 0 point-wise everywhere. That is:
\[
    \Delta u= 0 \qquad\text{(Laplace's Equation!)}
\] 
\end{proof}
\end{document}
