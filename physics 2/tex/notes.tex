\section{Vectors and Distance}
\emph{Vector Addition}
$$u=a\hat{i}+b\hat{j}, v=c\hat{i}+d\hat{j}$$
$$u+v=(a+c)\hat{i}+(b+d)\hat{j}$$
$$\tan{\theta}=\frac{u_y}{u_x}=\frac{b}{a}$$
\emph{Distance}
$$r_{ab}=\sqrt{(a_x-b_x)^2+(a_y-b_y)^2}$$
\section{Coulomb's Law}
$$\mathbf{F}_{12}(\hat{r}) = k\frac{q_1q_2}{r^2}\hat{r}$$
$$\mathbf{F_{\text{net}}} = \sum_{j=1}^N\mathbf{F_{ij}}$$
\section{Electric Field}
$$\mathbf{F}=q\mathbf{E}\implies \mathbf{E}=\frac{\mathbf{F}}{q}$$
For a point charge
$$\mathbf{E}=k\frac{q}{r^2}$$
$$\mathbf{E}_{\text{net}}=\sum_{i=1}^N\mathbf{E}_i$$
Electric field points towards negative charges, away from positive charges
\section{Continuous Charge Distributions}
\begin{table}[H]
    \scriptsize
    \centering
    \begin{tabular}{c|c|c|c}
        & Line & Area & Volume\\
        \hline
        Density &  $Q=\lambda L$ & $Q=\eta A$ & $Q=\rho V$\\
        \hline
        Units &  $Cm^{-1}$ & $Cm^{-2}$ & $Cm^{-3}$\\
        \hline
        Cont. &  $dq=\lambda dL$ & $dq = \eta dA$ & $dq = \rho dV$
    \end{tabular}
\end{table}
\section{Common Electric Fields}

Unif. inf. charged wire: $\frac{\lambda}{2r\pi\varepsilon_0}$\\
Unif. inf. charged plane: $\frac{\eta}{2\varepsilon_0}$\\
Unif. charged sphere ($r\geq R$): $k\frac{Q}{r^2}$\\
Unif. charged sphere ($r\leq R$): $k\frac{Qr}{R^3}$

\section{Flux}
"A measure of the number of field lines which pass through a surface"\\
\emph{Uniform field and plane}:
$$\Phi = \mathbf{E}\cdot \mathbf{A}=EA\cos{\theta}$$
Flux depends on the strength \mathbf{E} of the field, the area \mathbf{A} of the surface, and the angle $\theta$ between them.\\
\emph{Variable field/surface}:\\
$$d\phi=\mathbf{E}\cdot d\mathbf{A}\implies\phi=\int\mathbf{E}\cdot d\mathbf{A}$$
\section{Gauss' Law}
The flux through a \emph{closed} surface only depends on the charge \emph{inside} the surface, it doesn’t depend on the size or shape of the surface.\\
Flux doesn’t depend on charge \emph{distribution} inside a gaussian surface, it depends only on the \emph{total enclosed charge}.\\
The flux of external field through a closed surface is always 0. 
$$\oint\mathbf{E}\cdot d\mathbf{A}=\frac{Q_{\text{enc}}}{\varepsilon_0}$$

\emph{Example:}
Consider an infinite wire with linear charge density $\lambda C\cdot m^{-1}$. What is the electric field \emph{r} meters away from the wire?
\begin{figure}[H]
    \centering
    \includesvg[scale=.1]{./images/wire.svg}
    \caption{Gaussian surface around wire}
\end{figure}
$$\oint\mathbf{E}\cdot d\mathbf{A}=\frac{\lambda\cdot\ell}{\varepsilon_0}$$
$$E\cdot{2r\pi\ell}=\frac{\lambda\cdot\ell}{\varepsilon_0}\implies E=\frac{\lambda}{2r\pi\varepsilon_0}$$
\section{Conductors}
Electric field inside a conductor is 0 as it has reached equilibrium.\\
Charge inside a conductor is always zero, otherwise a field would be inside.\\
If a conductor is charged, all of its charges are on its surface.\\
Electric field right outside a conductor is perpendicular to its surface and has magnitude $\frac{\eta}{\varepsilon_0}$.
\subsection{Cavities}
\begin{figure}[H]
    \centering
    \includesvg[scale=.4]{./images/cavities.svg}
    \caption{Charge results from spherical charged cavities inside conducting sphere}
\end{figure}
\section{Electric Potential Energy}
Electric fields store \emph{potential energy}.\\
Field accelerates charge $\equiv$ Field does work on charge\\
Recall $W = \int_A^B\mathbf{F}\cdot d\mathbf{s}$. For conservative forces $W=K_f-K_i=U_i-U_f$.\\
The change in potential energy is $\Delta U=U_f-U-i=-W$\\
\emph{Constant force}:
$$W=\int_A^B\mathbf{F}\cdot d\mathbf{s}=\mathbf{F}\cdot\int_A^Bd\mathbf{s}=F\Delta r\cos\theta$$
\subsection{Uniform Electric Field}
A positive charge will speed up as it "falls" to the negative side in the field.\\
There is a constant force $F=qE$ in the direction of displacement. The work done is
$$W_{\text{elec}}=qEs_i-qEs_f$$
The change in \emph{electric potential energy} is
$$\Delta U_{\text{elec}}=-W_{\text{elec}}$$
A \emph{negatively charged} particle in the same field has negative potential energy.
\subsection{Potential Energy of Point Charges}
Consider two like charges $q_1$ and $q_2$. The electric field of $q_1$ pushes $q_2$. The work done is
$$W_{\text{elec}}=\int_{x_i}^{x_f}F_{12}dx=\int_{x_i}^{x_f}k\frac{q_1q_2}{x^2}dx$$
$$\frac{-kq_1q_2}{x}\vert_{x_i}^{x_f}=-kq_1q_2(\frac{1}{x_f}-\frac{1}{x_i})$$
The change in electric potential energy is
$$\Delta U_{\text{elec}}=-W_{\text{elec}}=kq_1q_2(\frac{1}{x_f}-\frac{1}{x_i})$$
Now assume $x_i=\infty$. Then the electric potential energy of two point charges becomes
$$U_{\text{elec}}=\frac{kq_1q_2}{x}$$
Electric field forces are conservative; their work done doesn't depend on path.\\
For a set of charges, potential energy is
$$U=k\sum_{i<j}\frac{q_iq_j}{r_{ij}}$$
\section{Electric Potential Difference}
\emph{Electric potential difference}: Change in potential energy upon 
displacement a unit charge in an electric field
$$\Delta V_{AB}=\frac{\Delta U_{AB}}{q}, \text{measured in volts:} V=\frac{J}{C}$$
$$\Delta V_{AB} = - \int_A^B\mathbf{E}\cdot d\mathbf{s}$$
\emph{Electrical potential in constant field}:
$$\Delta V_{AB}=V_B-V_A=-\frac{W_{AB}}{q}$$
$$=-\int_A^B\mathbf{E}\cdot d\mathbf{s}=-Es\cos{\theta}$$
\emph{Electric potential inside capacitor}:
$$\Delta V_C = V_+ - V_- = Ed$$
\emph{Electric potential for point charges}: The potential of a point charge $q$ at a distance $r$ is
$$V(r)=-\int_\infty^r\mathbf{E}\cdot d\mathbf{r} = -\int_\infty^r\frac{kq}{r^2}dr=\frac{kq}{r}$$
Superposition applies:
$$V_{\text{net}}=k\sum_{i=1}^N\frac{q_i}{r_i}$$
\emph{Electric potential for a continuous charge distribution}:
$$dV=k\frac{dq}{r}\implies V=k\int\frac{dq}{r}$$
\subsection{Potential from Electric Field}
$$V(b) - V(a)=-\int_a^b\mathbf{E}\cdot d\mathbf{l}$$
\emph{Example}:
Consider a capacitor
\begin{figure}[H]
    \centering
    \includesvg[scale=.3]{./images/capacitor.svg}
    \caption{Capacitor}
\end{figure}
Let $V(0)$ be at the negative plate, so $V(0)=0$. Then
$$V(s)-V(0)=V(s)=-\int_0^s\mathbf{E}\cdot d\mathbf{x}$$
$$E=-\frac{Q}{\varepsilon_0A},\mathbf{E}\cdot d\mathbf{x}=-Edx$$
$$\therefore V(s)=-(-\frac{Q}{\varepsilon_0A})\int_0^sdx=Es$$
\subsection{Electric Field from Potential}
$$E_x=-\frac{dV}{dx}, E_y=-\frac{dV}{dy},...$$
$$\Vec{E}=-\nabla V=-(\frac{\partial V}{\partial x}\hat{i}+\frac{\partial V}{\partial y}\hat{j}+\frac{\partial V}{\partial z}\hat{k})$$
Flux doesn’t depend on charge \emph{distribution} inside a gaussian surface, it depends only on the \emph{total enclosed charge}.\\
The flux of external field through a closed surface is always 0. 
$$\oint\mathbf{E}\cdot d\mathbf{A}=\frac{Q_{\text{enc}}}{\varepsilon_0}$$

\emph{Example:}
Consider an infinite wire with linear charge density $\lambda C\cdot m^{-1}$. What is the electric field \emph{r} meters away from the wire?
\begin{figure}[H]
    \centering
    \includesvg[scale=.1]{./images/wire.svg}
    \caption{Gaussian surface around wire}
\end{figure}
$$\oint\mathbf{E}\cdot d\mathbf{A}=\frac{\lambda\cdot\ell}{\varepsilon_0}$$
$$E\cdot{2r\pi\ell}=\frac{\lambda\cdot\ell}{\varepsilon_0}\implies E=\frac{\lambda}{2r\pi\varepsilon_0}$$
\section{Conductors}
Electric field inside a conductor is 0 as it has reached equilibrium.\\
Charge inside a conductor is always zero, otherwise a field would be inside.\\
If a conductor is charged, all of its charges are on its surface.\\
Electric field right outside a conductor is perpendicular to its surface and has magnitude $\frac{\eta}{\varepsilon_0}$.
\subsection{Cavities}
\begin{figure}[H]
    \centering
    \includesvg[scale=.4]{./images/cavities.svg}
    \caption{Charge results from spherical charged cavities inside conducting sphere}
\end{figure}
\section{Electric Potential Energy}
Electric fields store \emph{potential energy}.\\
Field accelerates charge $\equiv$ Field does work on charge\\
Recall $W = \int_A^B\mathbf{F}\cdot d\mathbf{s}$. For conservative forces $W=K_f-K-i=U_i-U_f$.\\
The change in potential energy is $\Delta U=U_f-U-i=-W$\\
\emph{Constant force}:
$$W=\int_A^B\mathbf{F}\cdot d\mathbf{s}=\mathbf{F}\cdot\int_A^Bd\mathbf{s}=F\Delta r\cos\theta$$
\subsection{Uniform Electric Field}
A positive charge will speed up as it "falls" to the negative side in the field.\\
There is a constant force $F=qE$ in the direction of displacement. The work done is
$$W_{\text{elec}}=qEs_i-qEs_f$$
The change in \emph{electric potential energy} is
$$\Delta U_{\text{elec}}=-W_{\text{elec}}$$
A \emph{negatively charged} particle in the same field has negative potential energy.
\subsection{Potential Energy of Point Charges}
Consider two like charges $q_1$ and $q_2$. The electric field of $q_1$ pushes $q_2$. The work done is
$$W_{\text{elec}}=\int_{x_i}^{x_f}F_{12}dx=\int_{x_i}^{x_f}k\frac{q_1q_2}{x^2}dx$$
$$\frac{-kq_1q_2}{x}\vert_{x_i}^{x_f}=-kq_1q_2(\frac{1}{x_f}-\frac{1}{x_i})$$
The change in electric potential energy is
$$\Delta U_{\text{elec}}=-W_{\text{elec}}=kq_1q_2(\frac{1}{x_f}-\frac{1}{x_i})$$
Now assume $x_i=\infty$. Then the electric potential energy of two point charges becomes
$$U_{\text{elec}}=\frac{kq_1q_2}{x}$$
Electric field forces are conservative; their work done doesn't depend on path.\\
For a set of charges, potential energy is
$$U=k\sum_{i<j}\frac{q_iq_j}{r_{ij}}$$
\section{Electric Potential Difference}
\emph{Electric potential difference}: Change in potential energy upon 
displacement a unit charge in an electric field
$$\Delta V_{AB}=\frac{\Delta U_{AB}}{q}, \text{measured in volts:} V=\frac{J}{C}$$
$$\Delta V_{AB} = - \int_A^B\mathbf{E}\cdot d\mathbf{s}$$
\emph{Electrical potential in constant field}:
$$\Delta V_{AB}=V_B-V_A=-\frac{W_{AB}}{q}$$
$$=-\int_A^B\mathbf{E}\cdot d\mathbf{s}=-Es\cos{\theta}$$
\emph{Electric potential inside capacitor}:
$$\Delta V_C = V_+ - V_- = Ed$$
\emph{Electric potential for point charges}: The potential of a point charge $q$ at a distance $r$ is
$$V(r)=-\int_\infty^r\mathbf{E}\cdot d\mathbf{r} = -\int_\infty^r\frac{kq}{r^2}dr=\frac{kq}{r}$$
Superposition applies:
$$V_{\text{net}}=k\sum_{i=1}^N\frac{q_i}{r_i}$$
\emph{Electric potential for a continuous charge distribution}:
$$dV=k\frac{dq}{r}\implies V=k\int\frac{dq}{r}$$
\subsection{Potential from Electric Field}
$$V(b) - V(a)=-\int_a^b\mathbf{E}\cdot d\mathbf{l}$$
\emph{Example}:
Consider a capacitor
\begin{figure}[H]
    \centering
    \includesvg[scale=.3]{./images/capacitor.svg}
    \caption{Capacitor}
\end{figure}
Let $V(0)$ be at the negative plate, so $V(0)=0$. Then
$$V(s)-V(0)=V(s)=-\int_0^s\mathbf{E}\cdot d\mathbf{x}$$
$$E=-\frac{Q}{\varepsilon_0A},\mathbf{E}\cdot d\mathbf{x}=-Edx$$
$$\therefore V(s)=-(-\frac{Q}{\varepsilon_0A})\int_0^sdx=Es$$
\subsection{Electric Field from Potential}
$$E_x=-\frac{dV}{dx}, E_y=-\frac{dV}{dy},...$$
$$\Vec{E}=-\nabla V=-(\frac{\partial V}{\partial x}\hat{i}+\frac{\partial V}{\partial y}\hat{j}+\frac{\partial V}{\partial z}\hat{k})$$
